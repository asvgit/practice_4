\stepcounter{mysection}\section{\arabic{mysection} Теоретическая часть}

\subsection{Позитивно-образованные формулы}

Позитивно-образованными формулами (ПО-формылами, ПОФ) называется вид формул, для записи которых используются только позитивные типовые кванторы \forall{}  и \exists:

Пусть X — множество переменных, и A — конъюнкт.

1. $\exists_x A$ и $\forall_x A$ есть \exists-ПОФ и \forall-ПОФ соответственно.

2. Если F = \{$F_1$,…,$F_n$\} есть \forall-ПОФы, тогда  ∶ F есть \exists-ПОФ.

3. Если F = \{$F_1$,…,$F_n$\} есть \exists-ПОФы, тогда  ∶ F есть \forall-ПОФ.

4. Любая \exists-ПОФ или \forall-ПОФ есть ПОФ.

Данные формулы не содержат операторов отрицания. Также ПО-формула является особым видом записи классических формул языка предикатов, подобно КНФ, ДНФ и др., поскольку любая формула языка предикатов первого порядка представима как позитивно-образованная формула.

Канонический вид ПО-формулы начинается с \forall\emptyset. Очевидно, что любая ПОФ приводима к каноническому виду. 

Типовые кванторы \forall\emptyset и \exists\emptyset называются фиктивными, поскольку не влияют на истинность формулы и не связывают никаких переменных, а только лишь служат конструкциями сохраняющими корректную запись ПО-формулы.

Для удобства ПО-формулы представляются в древовидной форме:

$Q_x A: $\{$F_1,...,F_n$\}$ \equiv Q_x A:  $
\begin{math}
    \begin{cases}
        F_1\\
        ...\\
        F_n
    \end{cases},
\end{math}

 где Fi – ПО-формула, А – набор атомарных формул, Q некоторый квантор, который отличен от кванторов в начале формул F.

 Некоторые части канонической ПО-формулы имеют специальные названия:

 1. Корневой узел \forall\emptyset называется корнем ПО-формулы;

 2. Дочерние узлы корня ПО-формулы имеют вид $\exists_x A$ и называются базами ПО-формулы, конъюнкт А называется базой фактов, а вся подформула начинающаяся с базового узла называется базовой подформулой;

 3. Дочерние узлы баз имею вид $\forall_x B$ и называются вопросами к родительской базе. Если вопрос является листовым узлом $\forall_x B\equiv\forall_x B:false$, то он называется целевым вопросом.

 4. Поддеревья вопросов называются консеквентами или следствиями. Следствием целевого вопроса является false.


\newpage
\subsection{Пример соответствия ПОВ и програмы на языке Prolog}
Рассмотрим следующую программу на языке Пролог и запрос к ней:
\\
\texttt{\raggedright\noindent
\\
in(a,b).\\
in(b,c).\\
it(X,Y):-in(X,Z),in(Z,Y).\\
?- it(a,X).\\
}

Соответствующая данной программе ПО--формула: \\

$\forall \emptyset\colon\exists in(a,b),in(b,c)\colon\left\{
\begin{array}{lcl}
 \forall_{x,y,z}in(x,z),in(z,y)\colon\exists it(x,y) \\
 \forall_x it(a,x)
\end{array}
\right..$ \\



Этот пример очень важен и будет упомянут далее.
