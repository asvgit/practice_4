\stepcounter{mysection}\section{\arabic{mysection} Основная часть}

	Целью данного исследования является выявление принципиальной возможности и целесообразности построения и использования системы логического вывода в рассматриваемой задаче. Поэтому оправданы все упрощения, которые неизбежны при построении моделей и переходе к реальному приложению.

Первым основным упрощение является дискретность времени с заданной величиной интервала между соседними моментами времени (тактами), содержащее начальный момент и бесконечно продолжимое вправо. Высота этажей считается одинаковой, и скорость движения лифта с одного этажа на другой полагаем равной одному такту. Длительность остановки кабин для входа-выхода пассажиров равняется одному такту. Так же не рассматриваются случаи переполнения кабин.

Считается, что любой пассажир придерживается следующим правилам:

1) Для вызова лифта он нажимает на этаже кнопку вызова и ждёт кабину, без ложных и ошибочных вызовов;

2) Войдя в кабину, пассажир задаёт ей команду, для чего он нажимает кнопку нужного этажа, который вносится в маршрут данной кабины, без ложных и ошибочных команд.

Простейшим алгоритмом принятия решения является поиск ближайшей кабины к месту вызова. Однако, термин «ближайшая» требует уточнения и рассмотрения примера. 

Допустим, есть система из k = 2 кабин, способных перемещаться по n = 5 этажам. Пусть кабины находятся на 1-м и 2-м этажах, первая пуста и находится в покои, а второй предстоят остановки на 3-м и 4-м этажах. Поступает вызов с пятого этажа, и первая кабина получается ближайшей, так как её требуется 4 такта, а второй кабине требуется 5 тактов. Получается дистанция – это количество тактов, которое необходимо кабине, чтобы добраться до этажа, выполняя уже сформированный маршрут (рисунок 1).

Есть и другой подход, который основывается на исключении худших альтернатив на основе логического вывода. И если после сокращения допустимых альтернатив их останется несколько, то выбор может быть случайным или основываться на каких-либо критериях. В этом и предыдущих подходах одним из основных критериев является средняя длительность ожиданий.

Основными объектами в данной модели являются кабина cab человек man. В момент времени t кабина имеет вид cab(i, e, S, t), где i – идентификатор кабины, e – этаж, а S - маршрут кабины, список этажей. Человек имеет вид man(e, d, τ, t), где e – этаж, d – целевой этаж, который добавляется в маршрут S в момент входа человека в кабину и d ≠ e, τ – длительность ожидания человеком кабины. Дистанцией же будет dist(e, S, d, i, α), где α – это дистанция от кабины i на этаже е с маршрутом S до этажа d, где произошёл вызов.
