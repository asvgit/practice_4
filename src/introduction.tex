\section{Задание на практику}
	В результате прохождения практики необходимо построить математическую модель.
		Сделать описание основных её компонентов, разработать правила взаимодействия 
		элементов модели.

	В ходе практики должны быть освоены компетенции:
		\begin{itemize}
			\item способность анализировать профессиональную информацию, выделять в ней главное, структурировать, оформлять и представлять в виде аналитических обзоров с обоснованными выводами и рекомендациями
			\item способность проектировать вспомогательные и специализированные языки программирования и языки представления данных;
			\item способность проектировать распределенные информационные системы, их компоненты и протоколы их взаимодействия.
		\end{itemize}

\newpage
\section{Введение}

Автоматизированные системы управления активно приходят в повседневную жизнь человечества. Сначала, это были системы для управления производственным процессом на крупных предприятиях, теперь данные системы решают и бытовые задачи. Одной из таких задач является доставка человека с одного этажа на другой. Данная задача достаточно подробно описана в книге С.В. Васильева [1], там имеются абстрактная модель, логическая модель и логический вывод.

С предложенным логическим выводом справилась бы система автоматического доказательства теорем А.А. Ларионова [2], но реализация всей системы лифтов на позитивно образованных формулах требует больших трудозатрат и будет носить чисто исследовательский характер.
